\chapter{緒論 \label{CH: intro}}

\hspace*{8mm} 在加速壽命試驗(Accelerated Life Testing, ALT)研究中,實驗設計的主要目標通常是提高參數估計的精度,例如透過 $c$-optimal 設計來最小化參數估計的變異。然而,當存在多個候選模型時,僅關注參數估計可能不足以確保模型的正確性。我們關注的核心問題是:如何設計實驗,使所收集的數據能有效區分不同的加速壽命模型,進而選擇最能反映產品實際失效行為的模型?這正是模型辨識設計(Model Discrimination Design)所要解決的關鍵挑戰。

\hspace*{8mm} 可靠度試驗是產品壽命分析中的重要環節,而 ALT 是透過施加高於正常使用條件的應力(如溫度、濕度、震動等)來加速產品老化,使研究人員能在有限時間內觀察失效現象。多數 ALT 模型以 Arrhenius 模型為基礎,假設壽命與溫度呈指數關係。然而,在實際應用中,可能存在多個結構不同但皆合理的模型,導致無法單靠既有經驗選定最佳模型。因此,如何透過設計適當的實驗條件,以產生能夠有效區分模型的資料,是 ALT 設計的重要議題。另一方面,即使在加速條件下進行試驗,產品仍可能未於測試期間發生失效,導致資料呈現 Type I 設限(Type I censored)現象。此時,研究者需設定合理的設限時間(censoring time),以控制試驗成本與時程,同時確保資料仍具備足夠資訊進行分析。因此,在考慮模型辨識設計時,設限資料的特性亦須一併納入評估。

\hspace*{8mm} 雖然模型辨識設計在其他領域已有所研究,但在 ALT 中仍相對較少受到關注。例如,\cite{nasir2015simulation} 提出的模型辨識方法主要基於貝葉斯最佳化策略,透過 Hellinger 距離來評估模型區分能力。然而,該方法在計算成本與結果穩定性方面仍存在挑戰,且主要聚焦於特定測試條件,未廣泛考慮不同類型的距離衡量準則。因此,本研究希望進一步探索加速試驗下的模型辨識問題,並考慮 Type I 設限數據(censored data)對實驗設計的影響。

\hspace*{8mm} 此研究第一個主要貢獻,在於提出四種基於不同散度的模型辨識設計方法,分別為 CKL-、CLW-、CB- 與 C$\chi^2$-optimal,這些準則皆針對設限情境下的實驗資料進行調整,期望提供更具彈性與準確性的模型選擇依據。

\hspace*{8mm} 如何生成模型辨識是這項研究的另一個關鍵。由於涉及的最佳化問題是嵌套的,並且通常缺乏目標函數的封閉解,我們進一步提出了一種高效的混合搜索演算法,結合粒子群優化(PSO)和基於牛頓法的方法,例如 L-BFGS 算法,以提高計算效率和收斂穩定性。我們展示了這種方法如何在合理的計算時間內識別出具有高模型區分能力的實驗設計,並通過數值模擬驗證其性能。

\hspace*{8mm} 本論文的架構如下:章節 \ref{CH: review} 回顧相關文獻,探討模型辨識設計的發展與現有最佳化標準。章節 \ref{CH: method} 提出我們的設計準則與最佳化演算法。章節 \ref{CH: simulation} 展示數值實驗結果,並比較不同設計標準在不同情境下的表現。章節 \ref{CH: conclusion} 則總結主要發現並提出了可能的未來研究方向。