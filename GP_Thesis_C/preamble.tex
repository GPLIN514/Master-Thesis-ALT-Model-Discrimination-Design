\documentclass[12pt, a4paper, twoside]{book}

\setlength{\textwidth}{13cm} 

\usepackage{caption}	
\usepackage{adjustbox}
\usepackage[sf,small]{titlesec}

\usepackage{fontspec} 
\usepackage{xeCJK} 
\xeCJKsetup{AutoFakeBold=true, AutoFakeSlant=true}
\setCJKmainfont{TW-Kai}
\setCJKsansfont{TW-Kai} 
\setmainfont{Times New Roman}
\defaultfontfeatures{Mapping=tex-text} 		
\usepackage{xunicode} 							
\usepackage{xltxtra} 							
\usepackage{amsmath, amssymb}
\usepackage{enumerate}
\usepackage{amsthm, amsfonts} 					

% \usepackage{tikz-flowchart}
\usepackage{graphicx, float, wrapfig} 
\usepackage[outercaption]{sidecap} 			
\usepackage{array, booktabs}      
\usepackage{color, xcolor}
\usepackage{tcolorbox}
\usepackage{longtable}
\usepackage{colortbl}                          			
\usepackage{listings}		
\usepackage{pdfpages}
\lstset{
  language=R,
  basicstyle=\ttfamily\footnotesize,
  keywordstyle=\color{blue},
  emph={min,beta,q},
  emphstyle=\color{black},
  commentstyle=\color{gray},
  stringstyle=\color{red},
  backgroundcolor=\color{gray!10},
  breaklines=true,
  frame=single,
  columns=fullflexible
}
\renewcommand{\lstlistingname}{程式碼}

\usepackage[parfill]{parskip} 					
\usepackage{geometry} 							
\geometry{twoside, top=1.5in,bottom=1.5in,inner=1.5in,outer=1.5in}
\usepackage{threeparttable}
\usepackage{makecell} 							
\usepackage{textcomp}
\usepackage{comment}
\usepackage{subcaption}
\usepackage{booktabs}
\usepackage{natbib}							
\usepackage{makeidx}						
\usepackage[parfill]{parskip} 
\usepackage{url}                            
    \def\UrlFont{\rm}                       

\usepackage[colorlinks=true,linkcolor=blue,citecolor=blue,urlcolor=blue,linktoc=page]{hyperref}

\usepackage{fancyhdr}
	\pagestyle{fancy}
	\fancyhf{}                              
    \renewcommand{\headrulewidth}{0pt}      
    \setlength{\headheight}{25.0pt}

\usepackage{newpxtext}
\usepackage{color, xcolor}

\XeTeXlinebreaklocale "zh"             
\XeTeXlinebreakskip = 0pt plus 1pt     
\newcommand{\imgdir}{images/}					
\renewcommand{\tablename}{表}					
\renewcommand{\figurename}{圖}				
\renewcommand{\contentsname}{目~錄}
\renewcommand{\listfigurename}{圖目錄}
\renewcommand{\listtablename}{表目錄}
\renewcommand{\indexname}{索引}
\renewcommand{\bibname}{參考文獻}


% 判斷是否為附錄章節,切換章節名稱格式
\usepackage{etoolbox}
\usepackage{zhnumber}
\usepackage{titlesec}
\newif\ifappendixmode
\appendixmodefalse

\pretocmd{\appendix}{\appendixmodetrue}{}{}

% 動態改變章節格式:附錄 vs 一般章節
\usepackage{titlesec}
\titleformat{\chapter}[display]
  {\raggedright\huge\bfseries}
  {\ifappendixmode 附錄\thechapter \else 第\ \zhnumber{\arabic{chapter}}\ 章 \fi}
  {0.2cm}{}

\theoremstyle{plain}
\newtheorem{de}{Definition}[section]		
\newtheorem{thm}{Theorem}[section]		
\newtheorem{lemma}[thm]{Lemma}			
\newtheorem{ex}{{\E Example}}				
\newtheorem{cor}{Corollary}[section]		
\newtheorem{exercise}{EXERCISE}			
\newtheorem{re}{\emph{Result}}[section]	
\newtheorem{axiom}{AXIOM}				
\renewcommand{\proofname}{\bf{Proof}}	
\newtheorem{conjecture}{假說}


\newcounter{quiz}						
\setcounter{quiz}{1}						

\parindent=0pt
\setcounter{tocdepth}{0}

\definecolor{bostonred}{rgb}{0.8, 0.0, 0.0}
\definecolor{PowerPointGreen}{RGB}{0,140,60}
\usepackage{mathtools,eqparbox}

\newcommand{\indices}[2]{{% \indices{<rows>}{<columns>}
  \begin{array}{@{}r@{}}
    \scriptstyle #2~\smash{\eqmakebox[ind]{$\scriptstyle\rightarrow$}} \\[-\jot]  
    \scriptstyle #1~\smash{\eqmakebox[ind]{$\scriptstyle\downarrow$}}
  \end{array}}}

\usepackage{multirow}
\newcommand\MyBox[2]{
    \fbox{\lower0.75cm
        \vbox to 1.7cm{\vfil
          \hbox to 1.7cm{\hfil\parbox{1.4cm}{#1\\#2}\hfil}
          \vfil}
    }
}

\usepackage{amsthm}  % 確保加載 amsthm 套件
\usepackage{amsmath, amssymb}  % 讓數學符號可以使用
% 定義 Theorem 環境
\newtheorem{theorem}{定理}

