\chapter{結論與未來研究方向\label{CH: conclusion}}

\section{研究結論與貢獻}

\hspace*{8mm} 本論文聚焦於可靠度模型下之模型辨識設計問題,與過往以貝氏方法結合抽樣策略為主之方法不同,本文所提出的方法具備更佳之計算效率與可重現性。傳統貝氏策略雖具理論完整性,然而在實務上常因運算時間冗長且結果不穩定而受限。為解決此問題,我們專注於近似設計,並採用文獻中廣泛認可的粒子群優化(PSO)演算法作為我們優化框架的核心。為了更好地適應加速壽命試驗中常見的 Type I 設限數據,我們提出了四個基於散度的設計標準:CKL-、CLW-、CB- 與 C$\chi^2$-optimal 設計準則,此外,發現到可靠度模型中的變異和離散結構可能取決於應力水準,我們進一步修改了 \cite{chen2020hybrid} 中的 PSO-QN 演算法,允許將應力相關變異設置納入目標函數評估。附錄 \ref{appendixB} 中提供了詳細的用戶手冊和新 PSO-QN 演算法的配置。在附錄 \ref{appendixC} 中,我們還使用 Shiny 開發了一個方便使用者使用的應用程式,以更好地描述我們的工作。總體而言,所提出的最佳化方法和改良準則,可為未來可靠度試驗設計提供更穩定且可擴展之演算法解決方案。

\hspace*{8mm} 數值研究部分首先以重現文獻結果為出發點,驗證本研究所採用方法之可行性。原始文獻透過封閉解形式的目標函數求得最佳設計,而本研究則改以數值積分方式進行,結果顯示兩者所獲得之設計極為接近,證實本方法具備良好之穩定性與正確性。接續模擬則聚焦於本研究核心——Arrhenius 模型,該模型為可靠度領域中常見之加速壽命模型。在給定真實模型與對立模型之參數數值,並考慮具 Type I 設限資料的情形,分別使用四種所提之散度(CKL、CLW、CB、C$\chi^2$)進行設計搜尋,結果顯示 CKL-optimal 準則於大多數情境中表現最佳。基於此發現,後續數值結果即聚焦於 CKL 散度,並進一步探討對立模型變異數未知、需納入參數估計的情況。最後,本研究亦考量真實世界中變異數可能隨應力改變而非為常數的情境,設計數個具代表性的模擬案例,以初步觀察此類情境下設計演算法的穩定性與挑戰,並指出未來在此方向仍有待深入研究與改良。

\hspace*{8mm} 綜合本研究之發現,我們成功整合多種散度與數值積分技巧,搭配混合式演算法進行模型辨識設計,提供一套適用於 Type I 設限可靠度資料的穩定設計流程。透過數值結果,我們發現 CKL-optimal 設計在多數情境下展現出良好表現,顯示該測度在可靠度模型辨識中的潛力。

\hspace*{8mm} 此外,本研究亦首度針對「變異數隨應力變化」的假設提出模型辨識設計架構,此類設定在可靠度領域具高度實務意涵,但過去文獻中幾無系統性探討。雖然初步數值結果尚未獲得理想辨識效果,但已揭示此類問題的理論與計算挑戰,並奠定未來演算法改進與方法延伸的研究基礎。整體而言,本研究除補足既有方法論於高彈性設計應用之不足,亦為探討複雜模型下的可靠度試驗設計提供新視角與實作方向。

\section{未來研究建議}

\hspace*{8mm} 儘管本研究所提出之數值最佳化設計流程已具備可行性與穩定性,實務應用中仍面臨若干挑戰,未來可從以下三方面進行延伸與改進:

\hspace*{8mm} 首先,在數值積分過程中,若被積函數中包含對數項(如 $\log(1 - F(y))$),且於尾端區域機率極小時,容易導致非有限值(Non-finite Values)或數值溢位,進而使積分無法正確計算。未來研究可考慮針對此類 Type I 設限項的數學形式進行修正,例如使用近似公式取代原始對數寫法,或透過改良積分範圍與策略,以提升數值穩定性與整體計算可靠性。

\hspace*{8mm} 其次,本研究目前假設內層最佳化問題的目標函數為可微分函數,因此使用 L-BFGS 等梯度式方法進行求解。然而,實際上此類目標函數雖具可微性,但未必處處平滑,可能存在多個局部極值或函數起伏劇烈的情況,導致 L-BFGS 等傳統方法容易陷入非全域解。未來可進一步分析目標函數的結構特性,並評估是否適合引入次梯度法(Subgradient Methods)、非平滑最佳化方法(Nonsmooth Optimization)或啟發式演算法作為替代策略。

\hspace*{8mm} 最後,本研究中針對模型辨識所採用的設計策略,是在對立模型的參數空間中尋找使散度(如 KL 散度)最小的參數組合,並以此為基礎進行最小最大化設計。然而,目前所求得之參數組合未必為實際上模型差異最小且具代表性的設定,亦可能因估計誤差而導致辨識效能降低。因此,未來可考慮先進行參數層級的最佳化,再據此進行模型辨識設計。舉例而言,可將 CKL-optimal 設計與 D-optimal 設計結合,兼顧模型間可辨識性與參數可估性,以提升整體設計的實務效能與解釋力。