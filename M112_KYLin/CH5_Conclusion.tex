\chapter{Conclusion and Future Works\label{CH: conclusion}}

\section{Summary and Key Contributions}

\hspace*{8mm} This thesis focuses on model discrimination design in the context of reliability modeling and offers an alternative to traditional Bayesian and sampling-based approaches. While these conventional methods are theoretically sound, they often suffer from high computational cost and limited reproducibility. To address these limitations, we focus on the approximation design and adopt the Particle Swarm Optimization (PSO) algorithm—widely recognized in the literature—as the core of our optimization framework. To better accommodate the Type I censored data commonly observed in Accelerated Life Testing, we propose four divergence measures-based design criteria: CKL-, CLW-, CB-, and C$\chi^2$-optimal designs. Furthermore, recognizing that the variance and dispersion structures in reliability models may depend on stress levels, we further modify the PSO-QN algorithm in \cite{chen2020hybrid} by allowing the stress-dependent variance settings into the objective function evaluation. Detailed user manual and the configuration of the new PSO-QN algorithm are provided in Appendix \ref{appendixB}. We also develop a user-friendly app using Shiny to better describe our work in the appendix \ref{appendixC}.  Overall, the proposed optimization methodology and enhanced criteria provide a robust and scalable solution for model discrimination design in reliability experiments.

\hspace*{8mm} The numerical study begins by reproducing results from a reference study to validate the feasibility of our proposed approach. While the original work obtains optimal designs through closed-form expression of the objective function, we adopt a numerical integration-based method. The results reveal that both approaches yield nearly identical designs, confirming the stability and correctness of our method. The subsequent numerical experiments center on the core of this research—the Arrhenius model, which is widely used in reliability studies. Under the assumption of known variance for both the true and rival models, and incorporating Type I censored data, we search for optimal designs using four proposed divergence measures criteria. The results suggest that CKL-optimal criterion consistently outperforms the others in most scenarios. Motivated by this, the following numerical results focus on the CKL divergence, extending the analysis to cases where the variance of the rival model is unknown and must be searched. Finally, to address practical situations where the variance structure may vary with stress level, we design several illustrative simulation cases to preliminarily examine the algorithm's behavior under such conditions, highlighting the challenges encountered and identifying directions for future refinement.

\hspace*{8mm} In summary, this study integrates multiple divergence measures with numerical integration techniques and a hybrid optimization framework to construct stable and effective model discrimination designs under Type I censored reliability data. Numerical results show that the CKL-optimal design consistently outperforms others in most scenarios, demonstrating its potential in model discrimination for reliability studies.

\hspace*{8mm} More importantly, this study is among the first to propose a design framework that considers stress-dependent variance structures in model discrimination. Although initial numerical results under such settings did not yield strong performance, they reveal critical challenges both theoretically and computationally. This opens a promising direction for future research to enhance algorithmic robustness and design strategies under complex modeling assumptions. Overall, this work not only extends the scope of existing optimal design methodologies but also provides new insights and tools for more realistic and flexible reliability test planning.

\section{Limitations and Future Directions}

\hspace*{8mm} While the proposed optimization framework in this study demonstrates feasibility and robustness, several limitations remain. Future research may address these challenges from the following three aspects:

\hspace*{8mm} First, during numerical integration, issues such as non-finite values or overflow may arise due to logarithmic terms in the integrand (e.g., $\log(1 - F(y))$), especially when the tail probability becomes extremely small. Future work may consider reformulating the censoring term using more stable approximations or optimizing the integration approach (e.g., adjusting bounds or resolution) to enhance numerical stability and computational reliability.

\hspace*{8mm} Second, while this study assumes that the inner-loop objective function is differentiable and therefore adopts gradient-based methods such as L-BFGS, differentiability alone does not guarantee smoothness. In practice, the objective function may exhibit irregular behavior—such as sharp curvature or multiple local extrema—which can mislead gradient-based solvers into premature convergence. Further investigation into the structure of the objective function is warranted, and alternatives such as subgradient methods, nonsmooth optimization techniques, or heuristic algorithms may be more robust under these conditions.

\hspace*{8mm} Finally, the current design strategy identifies the parameter combination in the rival model that minimizes divergence (e.g., KL divergence) from the true model and constructs the optimal design based on this configuration. However, the resulting parameter values may not be the most representative or precise due to estimation uncertainty, potentially limiting the effectiveness of the discrimination design. Future research may consider performing parameter-level optimization prior to discrimination design. For example, a compound approach that combines CKL-optimal and D-optimal designs could simultaneously enhance model discriminability and parameter estimability, thereby improving the practical utility and interpretability of the resulting designs.