\chapter{Introduction \label{CH: intro}}

\hspace*{8mm} In the study of Accelerated Life Testing (ALT), the primary goal of experimental design has traditionally focused on improving the precision of parameter estimation—for instance, through the use of $c$-optimal design to minimize estimation variance. However, when multiple candidate life models are considered, focusing solely on parameter estimation may not suffice to ensure model adequacy. The core question addressed in this research is: how can an experiment be designed such that the data collected effectively discriminate among competing models and allow the selection of the one that best describes product failure behavior? This is the key challenge that model discrimination design aims to resolve.

\hspace*{8mm} Reliability testing plays a crucial role in product life analysis, and ALT accelerates the observation of failure events by applying stress conditions beyond normal usage (e.g., temperature, humidity, vibration). Most ALT models are based on the Arrhenius equation, which assumes an exponential relationship between stress (typically temperature) and failure time. However, in real-world applications, multiple structurally distinct models may appear plausible. As the actual failure mechanism is not directly observable, designing experiments that help distinguish between these models becomes essential. Another practical issue arises from censoring. Even under accelerated stress conditions, some products may not fail within the study duration. As a result, a censoring time must be defined to limit the test duration while ensuring that the data remain informative. Hence, model discrimination design in ALT must also account for the presence of Type I censored observations when constructing design strategies.

\hspace*{8mm} Although model discrimination has been widely studied in other areas of statistics, its application in ALT remains limited. For instance, \cite{nasir2015simulation} proposed a Bayesian-based model discrimination strategy using the Hellinger distance to measure model separability. However, their approach faces challenges in computational cost and result stability, and is limited to specific scenarios and criteria. To address this gap, we aim to extend model discrimination design within the ALT framework by incorporating the effects of Type I censored data.

\hspace*{8mm} Our first major contribution to this research is the proposal of four model discrimination design criteria—CKL-, CLW-, CB-, and C$\chi^2$-optimal criteria—each based on a different divergence measures and adapted for use with Type I censored data. These criteria aim to improve the flexibility and accuracy of model selection under varied experimental conditions.

\hspace*{8mm} How to generate a model discrimination is another key to this research. Since the optimization problems involved are nested and generally lack closed-form expression of the objective function, we further propose an efficient hybrid search algorithm combining Particle Swarm Optimization (PSO) and the Newton-based approach such as the L-BFGS algorithm  to enhance computational efficiency and convergence stability. We demonstrate how this approach can identify experimental designs with high model discrimination power within a reasonable computation time and validate its performance through numerical simulations.

\hspace*{8mm} This thesis is organized as follows: Chapter \ref{CH: review} reviews relevant literature, discussing the development of model discrimination design and existing optimization criteria. Chapter \ref{CH: method} presents our proposed design criteria and optimization methodology. Chapter \ref{CH: simulation} presents numerical experiment results and compares the performance of different design criteria under various scenarios. Chapter \ref{CH: conclusion} summarizes key findings and suggests possible future research directions.